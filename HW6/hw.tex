\documentclass[11pt]{article}
\usepackage[utf8]{inputenc}
\usepackage[T1]{fontenc}
\usepackage[letterpaper]{geometry}
\geometry{verbose,tmargin=1.5cm,bmargin=1.5cm,lmargin=1.5cm,rmargin=1.5cm,columnsep=0.8cm}
\usepackage[english]{babel}
\usepackage{graphicx}
\usepackage{longtable}
\usepackage{float}
\usepackage{fouriernc}
\usepackage{amssymb}
\usepackage{amsmath}
\usepackage{hyperref}
\usepackage{framed}
\tolerance=1000
\usepackage{enumitem}
\usepackage{multicol}

\usepackage{tikz}
\usetikzlibrary{arrows,automata,positioning,fit,shapes}
 
\begin{document}

\begin{center}
\huge EECE7398 --- Compilers

Assignment 6

\vskip1cm

\normalsize\ttfamily Bruno S M M Morais <soutomaiormunizmo.b@husky.neu.edu>

All code can be found on \url{https://github.com/brunosmmm/eece7398}

\end{center}

\vskip0.5cm

\section*{Question 1}
\subsection*{item (a)}

The \textsc{First} and \textsc{Follow} set for the nonterminals:

\begin{multicols}{2}

\begin{table}[H]
\centering
\begin{tabular}{c | c}
 & \textsc{First} \\
\hline
F' & ty \\
F  & ty \\
A  & ty $\varepsilon$  \\
B  & , $\varepsilon$  \\
\hline
\end{tabular}
\end{table}

\columnbreak

\begin{table}[H]
\centering
\begin{tabular}{c | c}
 & \textsc{Follow} \\
\hline
F' & \$ \\
F  & \$ \\
A  & ) \\
B  & ) , \\
\hline
\end{tabular}
\end{table}

\end{multicols}

\subsection*{item (b)}

\begin{multicols}{2}

\begin{table}[H]
\centering
\begin{tabular}{ l c }
$I_{0}$ & \\
\hline
$F' \rightarrow \cdot F$ & $\xrightarrow{F} I_{1}$ \\
$F \rightarrow \cdot ty \; id \; \left( A\right)$ & $\xrightarrow{ty} I_{2}$ \\
\end{tabular}
\end{table}

\begin{table}[H]
\centering
\begin{tabular}{l c}
$I_{1}$ & \\
\hline
$F' \rightarrow F \cdot$ \\
\end{tabular}
\end{table}

\begin{table}[H]
\centering
\begin{tabular}{l c}
$I_{2}$ & \\
\hline
$F \rightarrow ty \cdot \; id \; \left( A\right)$ & $\xrightarrow{id} I_{3}$    
\end{tabular}
\end{table}

\begin{table}[H]
  \centering
  \begin{tabular}{l c}
    $I_{3}$ & \\
\hline
$F \rightarrow ty \; id \cdot \; \left( A\right)$ & $\xrightarrow{(} I_{4}$
  \end{tabular}
\end{table}

\begin{table}[H]
\centering
\begin{tabular}{l c}
$I_{4}$ & \\
\hline
$F \rightarrow ty \; id \; \left( \cdot A\right)$ & $\xrightarrow{A} I_{5}$ \\
$A \rightarrow \cdot ty \; id \; B$ & $\xrightarrow{ty} I_{6}$ \\
$A \rightarrow \varepsilon$    
\end{tabular}
\end{table}

\begin{table}[H]
\centering
\begin{tabular}{l c}
$I_{5}$ & \\
\hline
$F \rightarrow ty \; id \; \left( A \cdot \right)$ & $\xrightarrow{)} I_{7}$    
\end{tabular}
\end{table}

\begin{table}[H]
\centering
\begin{tabular}{l c}
$I_{6}$ & \\
\hline
$A \rightarrow ty \cdot \; id \; B$ & $\xrightarrow{id} I_{8}$    
\end{tabular}
\end{table}

\begin{table}[H]
\centering
\begin{tabular}{l c}
$I_{7}$ & \\
\hline
$F \rightarrow ty \; id \; \left( A\right) \cdot$ &     
\end{tabular}
\end{table}

\begin{table}[H]
\centering
\begin{tabular}{l c}
$I_{8}$ & \\
\hline
$A \rightarrow ty \; id \cdot \; B$ & $\xrightarrow{B} I_{9}$ \\
$B \rightarrow \cdot , \; ty \; id \; B$ & $\xrightarrow{,} I_{10}$ \\
$B \rightarrow \varepsilon$ & 
\end{tabular}
\end{table}

\begin{table}[H]
\centering
\begin{tabular}{l c}
$I_{9}$ & \\
\hline
$A \rightarrow ty \; id \; B \cdot$ &     
\end{tabular}
\end{table}

\begin{table}[H]
\centering
\begin{tabular}{l c}
$I_{10}$ & \\
\hline
$B \rightarrow , \; \cdot ty \; id \; B$ & $\xrightarrow{ty} I_{11}$    
\end{tabular}
\end{table}

\begin{table}[H]
\centering
\begin{tabular}{l c}
$I_{11}$ & \\
\hline
$B \rightarrow , \; ty \cdot \; id \; B$ & $\xrightarrow{id} I_{12}$    
\end{tabular}
\end{table}

\begin{table}[H]
\centering
\begin{tabular}{l c}
$I_{12}$ & \\
\hline
$B \rightarrow , \; ty \; id \cdot \; B$ & $\xrightarrow{B} I_{13}$ \\
$B \rightarrow \cdot , \; ty \; id \; B$ & $\xrightarrow{,} I_{10}$ \\
$B \rightarrow \varepsilon$ &    
\end{tabular}
\end{table}

\begin{table}[H]
\centering
\begin{tabular}{l c}
$I_{13}$ & \\
\hline
$B \rightarrow ty , \; \; id \; B \cdot$ &    
\end{tabular}
\end{table}

\end{multicols}

\subsection*{item (c)}

\begin{table}[H]
\centering
\begin{tabular}{c | c c c c c c | c c c}
State & ty  & id  & (  & )  & ,   & \$     & F & A & B  \\
\hline
0     & S2  &     &    &    &     &        & 1 &   &    \\
1     &     &     &    &    &     & Accept &   &   &    \\
2     &     & S3  &    &    &     &        &   &   &    \\
3     &     &     & S4 &    &     &        &   &   &    \\
4     & S6  &     &    & R3 &     &        &   & 5 &    \\
5     &     &     &    & S7 &     &        &   &   &    \\
6     &     & S8  &    &    &     &        &   &   &    \\
7     &     &     &    &    &     & R1     &   &   &    \\
8     &     &     &    & R5 & S10 &        &   &   & 9  \\
9     &     &     &    & R2 &     &        &   &   &    \\
10    & S11 &     &    &    &     &        &   &   &    \\
11    &     & S12 &    &    &     &        &   &   &    \\
12    &     &     &    & R5 & S10 &        &   &   & 13 \\
13    &     &     &    & R4 &     &        &   &   &    \\   

\end{tabular}
\end{table}

\subsection*{item (d)}

This is a SLR grammar because there are no shift / reduce conflicts in any state as seen when the table is constructed.

\end{document}
