\documentclass[11pt]{article}
\usepackage[utf8]{inputenc}
\usepackage[T1]{fontenc}
\usepackage[letterpaper]{geometry}
\geometry{verbose,tmargin=1.5cm,bmargin=1.5cm,lmargin=1.5cm,rmargin=1.5cm,columnsep=0.8cm}
\usepackage[english]{babel}
\usepackage{graphicx}
\usepackage{longtable}
\usepackage{float}
\usepackage{fouriernc}
\usepackage{amssymb}
\usepackage{amsmath}
\usepackage{hyperref}
\usepackage{framed}
\tolerance=1000
\usepackage{enumitem}
\usepackage{multicol}
\usepackage{minted}

\usepackage{tikz}
\usetikzlibrary{arrows,automata,positioning,fit,shapes}
 
\begin{document}

\begin{center}
\huge EECE7398 --- Compilers

Assignment 9

\vskip1cm

\normalsize\ttfamily Bruno S M M Morais <soutomaiormunizmo.b@husky.neu.edu>

All code can be found on \url{https://github.com/brunosmmm/eece7398}

\end{center}

\vskip0.5cm

\section*{Question 2}

LLVM output generated from source file:

\begin{minted}{llvm}
; ModuleID = 'TestModule'

define void @f() {
L0:
  %t0 = alloca i32
  %t1 = alloca i32
  %t2 = alloca i32
  %t3 = load i32* %t0
  %t4 = icmp sgt i32 %t3, 2
  br i1 %t4, label %L1, label %L2

L1:                                               ; preds = %L0
  %t5 = load i32* %t0
  %t6 = add i32 %t5, 1
  store i32 %t6, i32* %t1
  br label %L3

L2:                                               ; preds = %L0
  %t7 = load i32* %t2
  %t8 = icmp eq i32 %t7, 0
  br i1 %t8, label %L4, label %L5

L3:                                               ; preds = %L5, %L1
  store i32 1, i32* %t2
  ret void

L4:                                               ; preds = %L2
  store i32 0, i32* %t1
  br label %L5

L5:                                               ; preds = %L4, %L2
  br label %L3
}
\end{minted}

\pagebreak

Running opt with \texttt{cat test2.ll | opt -load ./reaching-def.so -reachingDef}, output:

\begin{verbatim}
Pass on function f
GEN for L5 = { }
GEN for L4 = { }
GEN for L3 = { }
GEN for L2 = { t7 t8 }
GEN for L1 = { t5 t6 }
GEN for L0 = { t0 t1 t2 t3 t4 }
Algorithm iterations: 3
Block L5: 
IN = { t0 t1 t2 t3 t4 t7 t8 }
OUT = { t0 t1 t2 t3 t4 t7 t8 }
Block L4: 
IN = { t0 t1 t2 t3 t4 t7 t8 }
OUT = { t0 t1 t2 t3 t4 t7 t8 }
Block L3: 
IN = { t0 t1 t2 t3 t4 t5 t6 t7 t8 }
OUT = { t0 t1 t2 t3 t4 t5 t6 t7 t8 }
Block L2: 
IN = { t0 t1 t2 t3 t4 }
OUT = { t0 t1 t2 t3 t4 t7 t8 }
Block L1: 
IN = { t0 t1 t2 t3 t4 }
OUT = { t0 t1 t2 t3 t4 t5 t6 }
Block L0: 
IN = { }
OUT = { t0 t1 t2 t3 t4 }
\end{verbatim}

The results observed are coherent with manual analysis of the LLVM code. In this case the algorithm goes through 3 iterations because there are nested control flow statements (\textsc{If}s) in the test code.

\end{document}
